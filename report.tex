\documentclass[12pt]{article}

\usepackage[margin=1in]{geometry} 
\usepackage{
	graphicx,		% For image modifications and the figure enviroment
	subcaption,     %for subfigures
	amsmath,		% For the AMS math styles
	amssymb,		% The extended AMS math symbol list
	amsthm,			% For use of theorems (works together with thmtools)
	fancyhdr,		% For fancy headers and footers on pages
	color,			% For handy color deafinitions (used cause of styling)
	xspace,			% Makes latex not eat spaces after commands
	hyperref,		% Makes links, references, the Table of Contents, etc. clickable.
	slashed
}
\usepackage{listings}
\usepackage{xcolor}

\newcommand{\N}{\mathbb{N}}
\newcommand{\Z}{\mathbb{Z}}
\newcommand{\R}{\mathbb{R}}
\renewcommand{\figurename}{Afbeelding}

\newenvironment{theorem}[2][Theorem]{\begin{trivlist}
		\item[\hskip \labelsep {\bfseries #1}\hskip \labelsep {\bfseries #2.}]}{\end{trivlist}}
\newenvironment{lemma}[2][Lemma]{\begin{trivlist}
		\item[\hskip \labelsep {\bfseries #1}\hskip \labelsep {\bfseries #2.}]}{\end{trivlist}}
\newenvironment{exercise}[2][Exercise]{\begin{trivlist}
		\item[\hskip \labelsep {\bfseries #1}\hskip \labelsep {\bfseries #2.}]}{\end{trivlist}}
\newenvironment{reflection}[2][Reflection]{\begin{trivlist}
		\item[\hskip \labelsep {\bfseries #1}\hskip \labelsep {\bfseries #2.}]}{\end{trivlist}}
\newenvironment{proposition}[2][Proposition]{\begin{trivlist}
		\item[\hskip \labelsep {\bfseries #1}\hskip \labelsep {\bfseries #2.}]}{\end{trivlist}}
\newenvironment{corollary}[2][Corollary]{\begin{trivlist}
		\item[\hskip \labelsep {\bfseries #1}\hskip \labelsep {\bfseries #2.}]}{\end{trivlist}}

\begin{document}
	
% --------------------------------------------------------------
%                         Start here
% --------------------------------------
	
\title{ADS research project, ordinal data}%replace X with the appropriate number
\author{F. Meijvis, D. Hessen, C. Spitoni}
\date{\today}
	
\maketitle

\section{Definition of ordinal regression}

The dependent random variable $Y$ has ordinal level of measurement. The number of possible values that $Y$ can take is $m+1$. A realization of $Y$ is denoted by $y$ and $y\in\{0,1,\ldots,m\}$. The conditional probability distribution of $Y$ given a set of interval predictors $\mathbf{x}=(x_{1},\ldots,x_{k})'$ is $P(Y=y\!\mid\!\mathbf{x})$. The regression of $Y$ on $\mathbf{x}$ is ordinal if and only if
\begin{equation}
	P(Y\leq y\!\mid\!\vec{x})=\sum_{t=0}^{y}P(Y=t\!\mid\!\vec{x}),\ \ \textup{for all $y$},
\end{equation}
is a monotonic function of $\mathbf{x}$. If higher values of $y$ are associated with higher values of $\mathbf{x}$, then this function should be non-increasing, for all $y$. If $P(Y\leq y\!\mid\!\mathbf{x})$ is a monotonic function of $x_{i}$, then its derivative with respect to $x_{i}$ is either non-negative or non-positive for all $x_{i}$.

\section{Multinomial logistic regression}
Let $z_{s}=1$ if $y=s$ and $z_{s}=0$ otherwise, for $s=1,\ldots,m$, then the conditional probability distribution of $Y$ given $\mathbf{x}$ can be written as
\begin{equation}
	P(Y=y\!\mid\!\vec{x})=P(Y=0\!\mid\!\vec{x})\prod_{s=1}^{m}\!\left\{\frac{P(Y=s\!\mid\!\vec{x})}{P(Y=0\!\mid\!\vec{x})}\right\}^{\!z_{s}},
\end{equation}
where
\begin{equation}
	P(Y=0\!\mid\!\vec{x})=\left\{1+\sum_{s=1}^{m}\frac{P(Y=s\!\mid\!\vec{x})}{P(Y=0\!\mid\!\vec{x})}\right\}^{\!-1}.
\end{equation}
In the multinomial logistic regression model,
\begin{equation}
	\frac{P(Y=s\!\mid\!\vec{x})}{P(Y=0\!\mid\!\vec{x})}=\textup{exp}(\alpha_{0s}+\vec{\alpha}'_{s}\vec{x}),\ \textup{for $s=1,\ldots,m$},
\end{equation}
where $\vec{\alpha}_{s}=(\alpha_{1s},\ldots,\alpha_{ks})'$, so that
\begin{equation}
	P(Y=y\!\mid\!\vec{x})=\frac{\textup{exp}\left\{\sum\limits_{s=1}^{m}z_{s}(\alpha_{0s}+\vec{\alpha}'_{s}\vec{x})\right\}}{1+\sum\limits_{s=1}^{m}\textup{exp}(\alpha_{0s}+\vec{\alpha}'_{s}\vec{x})}.
\end{equation}
It follows that
\begin{equation}
	P(Y\leq y\!\mid\!\vec{x})=\left\{\begin{array}{ll}\frac{1}{1+\sum\limits_{s=1}^{m}\textup{exp}(\alpha_{0s}+\vec{\alpha}'_{s}\vec{x})},&\textup{for $y=0$},\\
		\rule{0cm}{1cm}\frac{1+\sum\limits_{s=1}^{y}\textup{exp}(\alpha_{0s}+\vec{\alpha}'_{s}\vec{x})}{1+\sum\limits_{s=1}^{m}\textup{exp}(\alpha_{0s}+\vec{\alpha}'_{s}\vec{x})},&\textup{for $y>0$}.
	\end{array}\right.
\end{equation}
The multinomial logistic regression model has $(k+1)m$ parameters.

\section{conditions for $\alpha$}
In order to have the multinomial regression model satisfy the definition of ordinal regression, we require $P(Y \leq y \!\mid\! \vec{x})$ to be monotonic in $\vec{x}$. We will assume that higher values in $\vec{x}$ correspond to higher $y$. Furthermore, we treat both these statements are component wise.

For monotonicity, we require:
\begin{align*}
	\frac{\partial P(Y \leq y | \vec{x})}{\partial x_i} \leq 0
\end{align*}
For every $i$ and every $\vec{x}$. For class $y = 0$, this ultimately gives:
\begin{align*}
	\sum_{s = 1}^{m} \alpha_{is} \textup{exp} (\alpha_{0s} + \vec{\alpha_{s}}' \vec{x}) \geq 0
\end{align*}
This is satisfied if every $\alpha_{is} \geq 0$, but this may not be the strongest condition yet.

Note to self: I still want to test if it is possible to have 1 particular $\alpha \leq 0$

For clas $y > 0$, we find:
\begin{align*}
	(1 + \sum_{s = 1}^{m} \textup{exp} (\alpha_{0s} + \vec{\alpha_{s}}' \vec{x}))(\sum_{s = 1}^{y} \alpha_{is} \textup{exp} (\alpha_{0s} + \vec{\alpha_{s}}' \vec{x})) \\
	- (1 + \sum_{s = 1}^{y} \textup{exp} (\alpha_{0s} + \vec{\alpha_{s}}' \vec{x}))(\sum_{s = 1}^{m} \alpha_{is} \textup{exp} (\alpha_{0s} + \vec{\alpha_{s}}' \vec{x})) \leq 0
\end{align*}
Which reduces back to the previous equation when $y = 0$, and is trivially satisfied when $y = m$. 


\section{Adjacent categories model}
In this section we reparametrize the multinomial logistic regression model. In general we have:
\begin{align*}
	P(Y=y\!\mid\!\vec{x}) = P(Y=0\!\mid\!\vec{x}) \prod_{u = 1}^{y} \frac{P(Y=y\!\mid\!\vec{x})}{P(Y=y-1\!\mid\!\vec{x})}
\end{align*}
For $y = 1,\ldots,m$. In an adjacent categories model we assume
\begin{align*}
	\frac{P(Y=y\!\mid\!\vec{x})}{P(Y=y-1\!\mid\!\vec{x})} = \textup{exp}(\beta_{0u} + \vec{\beta_{u}}'\vec{x})
\end{align*}
Which can be seen as the probability of advancing from 1 category to the next (note that this requires an intrinsic ordering of the data i.e. ordinal data). Under these assumptions we find
\begin{align*}
	P(Y=y\!\mid\!\vec{x}) = P(Y=0\!\mid\!\vec{x}) \textup{exp}(\sum_{u = 1}^{y}[\beta_{0u} + \vec{\beta_{u}}'\vec{x}])
\end{align*}

By stating that $\vec{\beta}_{u}=(\beta_{1u},\ldots,\beta_{ku})'$, so that $\alpha_{0s}=\sum_{u=1}^{s}\beta_{0u}$ and $\vec{\alpha}_{s}=\sum_{u=1}^{s}\vec{\beta}_{u}$, we find
\begin{align*}
	P(Y=y\!\mid\!\vec{x})=\frac{\textup{exp}\!\left(\alpha_{0y}+y\vec{\beta}'\vec{x}\right)}{1+\sum\limits_{s=1}^{m}\textup{exp}(\alpha_{0s}+s\vec{\beta}'\vec{x})}
\end{align*}
Which reduces to
\begin{align*}
	P(Y=y\!\mid\!\vec{x})=\frac{\textup{exp}\!\left(\alpha_{0y}+\vec{\alpha_y}'\vec{x}\right)}{1+\sum\limits_{s=1}^{m}\textup{exp}(\alpha_{0s}+\vec{\alpha_s}'\vec{x})}
\end{align*}
by assuming the coefficients are independent of $s$ such that $\beta_{is}=\beta_{i}$, for $i=1,\ldots,k$ and all $s$, so that $\alpha_{is}=s\beta_{i}$.

\end{document}
